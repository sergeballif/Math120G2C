\documentclass{ximera}
%% handout
%% nohints
%% space
%% newpage
%% numbers

%% You can put user macros here
%% However, you cannot make new environments

\graphicspath{{./}{firstExample/}{secondExample/}}

\usepackage{url}
\usepackage{tikz}
\usepackage{tkz-euclide}
\usetkzobj{all}


\tikzstyle geometryDiagrams=[ultra thick,color=blue!50!black]
\pgfplotsset{compat=1.8}
  \usepackage[T1]{fontenc}
  \usepackage[utf8x]{inputenc} %% we can turn off input when making a master document

\prerequisites{none}
\outcomes{ximeraLatex}

\title{Function Notation}

\begin{document}
\begin{abstract}
We learn about function notation.
\end{abstract}
\maketitle

\section*{Recalling Function Notation}
We can think of a function as a machine that takes an input and returns an output. For example, at an arcade a quarter machine accepts dollars and gives out quarters. If $x$ is the number of dollars, and $Q$ is the number of quarters, then we write $Q(x)=4x$ to denote that when we input $x$ dollars we output $4x$ quarters.

In a nickel arcade the machine might accept $x$ dollars and give out $N$ nickels. In this case the function that governs the machine would be $N(x)=20x$. 

We usually give our functions names like $f(x)$ or $g(x)$ instead of just using the letter $y$. This is so we can keep track of which formula we are using. The notation is a good one. To evaluate a function such as $f(x)=x^2+4x-7$ we simply replace each $x$ with the same input value. For example,
\begin{itemize}
\item $f(0)=(0)^2+4(0)-7=-7$,
\item $f(1)=(1)^2+4(1)-7=-2$,
\item $f(-5)=(-5)^2+4(-5)-7=-2$.
\end{itemize}
As long as we replace each $x$ with the same input we will be fine. 

\begin{question}
Evaluate each value of the function $f(x)=x-\sqrt{x}$.
\begin{solution}
\begin{hint}
$f(9)=9-\sqrt{9}=9-3=6$. 
\end{hint}
$f(9)=$ $\answer{6}$.\\
$f(1)=$ $\answer{0}$.\\
$f(16)=$ $\answer{12}$.
\end{solution}
\end{question}

\begin{question}
Evaluate each value of the function $g(x)=\frac{x(x+1)(x-1)}{6}$.
\begin{solution}
\begin{hint}
$g(3)=\frac{(3)\left((3)+1\right)\left((3)-1\right)}{6}$. 
\end{hint}
$g(3)=$ $\answer{4}$.\\
$g(1)=$ $\answer{0}$.\\
$g(-2)=$ $\answer{-1}$.
\end{solution}
\end{question}


\section*{Technical Details}
In higher level math courses we care about the nitty gritty technical details. In this class a working understanding will suffice. We should know that definitions of the terms below.


A \emph{relation} from one set to another is a pairing of the elements in the first set with the elements in the second set. Thus a relation is a set of \emph{ordered pairs}.

A \emph{function} is a relation relating two sets called the \emph{domain} and \emph{range} in which each element of the domain is paired with exactly one element of the range. 

The domain is the set of input values, while the range is the set of output values.

For example $f(x)=x^2-1$ is a function.
\begin{itemize}
\item $f$ is the name of the function.
\item $x$ is the input variable or independent variable.
\item  $x^2-1$ is the output expression or dependent variable.
\item $f(x)$ denotes the ouput expression or dependent variable.
\item In set notation we would write $f=\{(x,x^2-1)\mid x\text{ is a real number}\}$.
\item The natural domain of $f$ is all real numbers.
\item The range of $f$ is $\{y\mid y\ge-1\}$.
\end{itemize}








\end{document}