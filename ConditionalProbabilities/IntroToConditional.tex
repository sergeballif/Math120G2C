\documentclass{ximera}
%% handout
%% nohints
%% space
%% newpage
%% numbers

%% You can put user macros here
%% However, you cannot make new environments

\graphicspath{{./}{firstExample/}{secondExample/}}

\usepackage{url}
\usepackage{tikz}
\usepackage{tkz-euclide}
\usetkzobj{all}


\tikzstyle geometryDiagrams=[ultra thick,color=blue!50!black]
\pgfplotsset{compat=1.8}
  \usepackage[T1]{fontenc}
  \usepackage[utf8x]{inputenc} %% we can turn off input when making a master document

\prerequisites{none}
\outcomes{ximeraLatex}

\title{Conditional Probability}

\begin{document}
\begin{abstract}
We introduce the idea of conditional probability and work through some simple examples.
\end{abstract}
\maketitle

\subsection*{Basic learning objectives}

These are the tasks you should be able to perform with reasonable fluency \textbf{when you arrive at our next class meeting}. Important new vocabulary words are indicated \emph{in italics}. 

\begin{itemize}
	\item Understand the definition of a \emph{conditional probability} and its relationship to the general concept of probability.
    \item Understand the basic notation of a general probability.
    \item Be able to compute a conditional probability.
\end{itemize}

\subsection*{Advanced learning objectives}

In addition to mastering the basic objectives, here are the tasks you should be able to perform \textbf{after class, with practice}: 

\begin{itemize}
	\item Conceptualize a conditional if-then statement in the context of a conditional probability.
    \item Apply the concept of conditional probability to the example of the false positive paradox.
\end{itemize}

When working with probabilities, we sometimes want to make certain types of restrictions to our sample space. This can sometimes lead to interesting situations that are not intuitive. The notation for conditional probabilities is the following:
\[ P(X | Y) = \text{``The probability that event $X$ occurs \emph{given than} event $Y$ occurred.''} \]

The best way to understand these situations is to think through a couple examples. Suppose that you are doing an experiment of rolling one red and one green 6-sided dice. The sample space contains 36 elements in the form (red, green):
\begin{image} \begin{tabular}{cccccc}
    (1,1) & (1,2) & (1,3) & (1,4) & (1,5) & (1,6) \\
    (2,1) & (2,2) & (2,3) & (2,4) & (2,5) & (2,6) \\
    (3,1) & (3,2) & (3,3) & (3,4) & (3,5) & (3,6) \\
    (4,1) & (4,2) & (4,3) & (4,4) & (4,5) & (4,6) \\
    (5,1) & (5,2) & (5,3) & (5,4) & (5,5) & (5,6) \\
    (6,1) & (6,2) & (6,3) & (6,4) & (6,5) & (6,6) \\
\end{tabular} \end{image}

This means that the probability of any particular dice roll is $\frac{1}{36}$ (where we match the dice colors to the appropriate number). With this sample space written out, we can ask the same types of questions as before and compute the result by counting:
\begin{itemize}
  \item What is the probability that the red die is a 4? (The red die is the first value.)
    \begin{image} \begin{tabular}{cccccc}
      (1,1) & (1,2) & (1,3) & (1,4) & (1,5) & (1,6) \\
      (2,1) & (2,2) & (2,3) & (2,4) & (2,5) & (2,6) \\
      (3,1) & (3,2) & (3,3) & (3,4) & (3,5) & (3,6) \\
      \cellcolor{lightgray}(4,1) & \cellcolor{lightgray}(4,2) & \cellcolor{lightgray}(4,3) & \cellcolor{lightgray}(4,4) & \cellcolor{lightgray}(4,5) & \cellcolor{lightgray}(4,6) \\
      (5,1) & (5,2) & (5,3) & (5,4) & (5,5) & (5,6) \\
      (6,1) & (6,2) & (6,3) & (6,4) & (6,5) & (6,6) \\
    \end{tabular} \end{image}
    Answer: $\frac{6}{36} = \frac{1}{6}$. (We should expect this since the dice are independent events!)
  \item What is the probability that the dice add up to 9?
    \begin{image} \begin{tabular}{cccccc}
      (1,1) & (1,2) & (1,3) & (1,4) & (1,5) & (1,6) \\
      (2,1) & (2,2) & (2,3) & (2,4) & (2,5) & (2,6) \\
      (3,1) & (3,2) & (3,3) & (3,4) & (3,5) & \cellcolor{lightgray}(3,6) \\
      (4,1) & (4,2) & (4,3) & (4,4) & \cellcolor{lightgray}(4,5) & (4,6) \\
      (5,1) & (5,2) & (5,3) & \cellcolor{lightgray}(5,4) & (5,5) & (5,6) \\
      (6,1) & (6,2) & \cellcolor{lightgray}(6,3) & (6,4) & (6,5) & (6,6) \\
    \end{tabular} \end{image}  
    Answer: $\frac{4}{36} = \frac{1}{9}$
\end{itemize}

Notice that in both of these problems, we did not include any conditions on the sample space. We simply took all possible combinations of dice rolls to get the 36 possibilities in the denominator. 

But now let's change the problem a bit. Suppose that we know that the total of the two dice is 9. What is the probability that a 1 was rolled? (Think for a moment!) You should quickly realize that it's impossible to get a total of 9 if you rolled a 1, so the probability of this happening is 0. This doesn't say that it's impossible to roll a 1, it's just impossible to roll a 1 \emph{given that} the total is 9.

Mathematically, the sample space has been reduced from the 36 combinations down to the 4 combinations shaded in the last example. And since 0 of those combinations have a 1 in them, the probability is expressed as $\frac{0}{4}$.

Now we will do a more interesting example. What is the probability of rolling a 3 given that the total is 6 or greater? First, we need to restrict the sample space to combinations that add up to 6 or greater:
    \begin{image} \begin{tabular}{cccccc}
      (1,1) & (1,2) & (1,3) & (1,4) & \cellcolor{lightgray}(1,5) & \cellcolor{lightgray}(1,6) \\
      (2,1) & (2,2) & (2,3) & \cellcolor{lightgray}(2,4) & \cellcolor{lightgray}(2,5) & \cellcolor{lightgray}(2,6) \\
      (3,1) & (3,2) & \cellcolor{lightgray}(3,3) & \cellcolor{lightgray}(3,4) & \cellcolor{lightgray}(3,5) & \cellcolor{lightgray}(3,6) \\
      (4,1) & \cellcolor{lightgray}(4,2) & \cellcolor{lightgray}(4,3) & \cellcolor{lightgray}(4,4) & \cellcolor{lightgray}(4,5) & \cellcolor{lightgray}(4,6) \\
      \cellcolor{lightgray}(5,1) & \cellcolor{lightgray}(5,2) & \cellcolor{lightgray}(5,3) & \cellcolor{lightgray}(5,4) & \cellcolor{lightgray}(5,5) & \cellcolor{lightgray}(5,6) \\
      \cellcolor{lightgray}(6,1) & \cellcolor{lightgray}(6,2) & \cellcolor{lightgray}(6,3) & \cellcolor{lightgray}(6,4) & \cellcolor{lightgray}(6,5) & \cellcolor{lightgray}(6,6) \\
    \end{tabular} \end{image}  
Within this restricted sample space, we need to count the events that have a 3 in them:
\begin{image} \begin{tabular}{cccccc}
     &  &  &  & (1,5) & (1,6) \\
     &  &  & (2,4) & (2,5) & (2,6) \\
     &  & \cellcolor{lightgray}(3,3) & \cellcolor{lightgray}(3,4) & \cellcolor{lightgray}(3,5) & \cellcolor{lightgray}(3,6) \\
     & (4,2) & \cellcolor{lightgray}(4,3) & (4,4) & (4,5) & (4,6) \\
    (5,1) & (5,2) & \cellcolor{lightgray}(5,3) & (5,4) & (5,5) & (5,6) \\
    (6,1) & (6,2) & \cellcolor{lightgray}(6,3) & (6,4) & (6,5) & (6,6) \\
\end{tabular} \end{image}
We can see that there are 7 combinations that contain a 3 and a total of 26 combinations that met the condition. This means that the probability is $\frac{7}{26} \approx 26.9\%$. (For comparison, the probability of rolling a 3 when rolling two dice is $\frac{11}{36} \approx 30.6\%$.)

This may not seem that complicated, but the next video will show some highly non-intuitive results that happen with conditional probabilities. Consider the following question: A man has two children. At least one of those children is a boy. What is the probability that the other child is also a boy? (Think for a moment!)

You probably guessed 50\%. And you would be wrong. But don't feel bad. This problem tricks almost everybody the first time. This problem is known as the boy-boy paradox (or sometimes the boy-girl paradox, if the question is asking for the probability that the other child is a girl). The following video will explain the paradox and give you more experience with conditional probabilities: \youtube{https://www.youtube.com/watch?v=RA6POO0x9h8}

\begin{question}
Which of the following are examples of dependent events?
  \begin{solution}
    \begin{multiple-choice}
      \choice[correct]{Removing two balls from an urn one at a time without replacement}
      \choice{Removing two balls from an urn one at a time, but replacing the ball after each draw}
      \choice{Flipping the same coin twice}
      \choice{Flipping two different coins}
      \choice[correct]{Dealing two cards from a standard deck of cards.}
      \choice[correct]{Dealing two cards from a deck of cards that contains no aces.}
      \end{multiple-choice}
    \begin{hint}
    Dependent events are when the outcome of one event has an influence on the outcome of the other event. Independent events are when the outcome of one event does not influence the outcome of the other event.
    \end{hint}
    \begin{hint}
    Imagine that you're performing the experiment and think about what happens after the first event. Is this connected to what is about to happen in the next part?
    \end{hint}
  \end{solution}
\end{question}

\begin{question}
The Gambler's Fallacy is the belief that...
  \begin{solution}
    \begin{multiple-choice}
      \choice{--- you can beat the house when you gamble.}
      \choice[correct]{--- the chances of future events  change based on previous events}
      \choice{--- superstitious acts can influence future outcomes}
      \end{multiple-choice}
    \begin{hint}
    The answer is in the video. The wording is slightly different, but the concept is the same.
    \end{hint}
  \end{solution}
\end{question}

\begin{question}
Which of the following are examples of the Gambler's Fallacy?
  \begin{solution}
    \begin{multiple-choice}
      \choice[correct]{The last 4 coin flips have been heads. We're due to see a tail!}
      \choice{Four aces have been dealt from this deck of cards. There's no way the next card will be an ace!}
      \choice[correct]{Our first child is going to be a girl because my mom's first child was a girl and my grandma's first child was a girl.}
      \end{multiple-choice}
  \end{solution}
\end{question}
