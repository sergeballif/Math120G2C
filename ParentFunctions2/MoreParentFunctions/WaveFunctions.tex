\documentclass{ximera}
%% handout
%% nohints
%% space
%% newpage
%% numbers

%% You can put user macros here
%% However, you cannot make new environments

\graphicspath{{./}{firstExample/}{secondExample/}}

\usepackage{url}
\usepackage{tikz}
\usepackage{tkz-euclide}
\usetkzobj{all}


\tikzstyle geometryDiagrams=[ultra thick,color=blue!50!black]
\pgfplotsset{compat=1.8}
  \usepackage[T1]{fontenc}
  \usepackage[utf8x]{inputenc} %% we can turn off input when making a master document

\prerequisites{none}
\outcomes{ximeraLatex}

\title{Wave Functions}

\begin{document}
\begin{abstract}
We meet the sine and cosine functions.
\end{abstract}
\maketitle


We start off by learning about two wave functions called the \emph{sine} and \emph{cosine} functions. In a trigonometry class you would learn a great deal about these functions. In this class we will just learn some of their important characteristics, and we will use them for some limited applications.

Open up \url{https://www.desmos.com/calculator/ubhracbupf} in a separate window. Watch the video at  -------------------- Answer each question below.

\begin{exercise}
Compute $\sin\left(\frac{\pi}{6}\right)$.
\begin{solution} 
\begin{hint}
Plug it into desmos or another calculator. 
\end{hint}
\begin{hint}
Be careful, because some calculators are in degree mode. You may need to find a mode setting to change the calculator to be in radian mode.
\end{hint}
$\sin\left(\frac{\pi}{6}\right)=\answer{.5}$.
\end{solution}
\end{exercise}

\begin{question}
The graph below shows a function of the form $f(x)=a\sin(bx)$, where $b\ge0$. Find the value of $a$ and $b$ to match the graph below.
\begin{image}
\includegraphics[width=9cm]{ParentFunctions2/MoreParentFunctions/mysterysine.png}
\end{image}
\begin{solution}
\begin{hint}
Use Desmos to create the function $f(x)$ with sliders for $a$ and $b$. 
\end{hint}
\begin{hint}
Remember that you can reach the menu to change the $x$-axis labels in desmos by clicking on the wrench.
\end{hint}
In the picture $a=\answer{-2}$ and $b=\answer{3}$.
\end{solution}
\end{question}

\begin{question}
In a function of the form $f(x)=a\sin(bx)$ the constant $a$ does what to the graph of $y=\sin(x)$?
  \begin{solution}
    \begin{multiple-choice}
      \choice{---shifts the graph to the right by $a$ units}
      \choice{---shifts the graph upward by $a$ units}
      \choice{---scales the graph horizontally by a factor of $a$}
      \choice[correct]{---scales the graph vertically by a factor of $a$}
    \end{multiple-choice}
    \begin{hint}
    Use Desmos to create the function $f(x)=a\sin(bx)$ with sliders for $a$ and $b$. As you change the value of $a$, what happens to the graph?
    \end{hint}
  \end{solution}
\end{question}

\begin{question}
In a function of the form $f(x)=a\sin(bx)$ the constant $b$ does what to the graph of $y=\sin(x)$?
  \begin{solution}
    \begin{multiple-choice}
      \choice{---shifts the graph to the right by $b$ units}
      \choice{---shifts the graph upward by $b$ units}
      \choice[correct]{---scales the graph horizontally by a factor of $\frac{1}{b}$}
      \choice{---scales the graph vertically by a factor of $\frac{1}{b}$}
    \end{multiple-choice}
    \begin{hint}
    Use Desmos to create the function $f(x)=a\sin(bx)$ with sliders for $a$ and $b$. As you change the value of $b$, what happens to the graph?
    \end{hint}
  \end{solution}
\end{question}

\end{document}