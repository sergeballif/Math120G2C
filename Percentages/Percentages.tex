\documentclass{ximera}
%% handout
%% nohints
%% space
%% newpage
%% numbers

%% You can put user macros here
%% However, you cannot make new environments

\graphicspath{{./}{secondExample/}}

\usepackage{tikz}
\usepackage{tkz-euclide}
\usetkzobj{all}

\tikzstyle geometryDiagrams=[ultra thick,color=blue!50!black]
 %% we can turn off input when making a master document

\prerequisites{none}
\outcomes{ximeraLatex}

\title{Calculations with percentages}

\begin{document}
\begin{abstract}
We review the concept of percentages and introducte \protect\url{ConsumerMath.org}.
\end{abstract}
\maketitle

\subsection*{Basic learning objectives}

These are the tasks you should be able to perform with reasonable fluency \textbf{when you arrive at our next class meeting}. Important new vocabulary words are indicated \emph{in italics}. 

\begin{itemize}
	\item Know how to compute any one of the whole, part, or percentage of a quantity given the other two.
    \item Be familiar with the basic purpose of the game at \url{ConsumerMath.org} and the role it will play in the course this semester.
\end{itemize}

\subsection*{Advanced learning objectives}

In addition to mastering the basic objectives, here are the tasks you should be able to perform \textbf{after class, with practice}: 

\begin{itemize}
	\item Be able to explain the related notions of \emph{percent change}, \emph{percent increase}, and \emph{percent decrease}.
	\item Understand the concept of a \emph{reference amount} and its impact on percentages.
	\item Be able to correctly identify reference amounts in problems involving quantities that change multiple times.
	\item Be able to compute percentages accurately with shifting reference amounts.
\end{itemize}

\end{document}